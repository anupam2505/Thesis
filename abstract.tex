\begin{abstract}

Field of cyber threats is evolving rapidly and every day multitude of new information
about malware and Advanced Persistent Threats (APTs) is generated by the
security professionals in the form of malware reports, blog articles, forum posts etc.
However, currently data received from different sources is examined and interpreted
by an analyst manually, a procedure that extremely obscures the practical use of this
data in an enterprise$'$s security infrastructure. Most of traditional approaches just
consider either one or two data sources for the generation of Indicators of Compromise
(IOCs) and misses some of the most valuable IOCs from other data sources.
This leaves some scope for attackers to bypass the defense systems. Additionally, current state of art lacks support to diverse IOC formats (e.g. STIX, OpenIOC, snort
etc.) which further reduce the effectiveness of these tools. Most importantly, lots
of Threat Intelligence (TI) tools are generating IOCs directly using regex expression
without understanding the contextual meaning of those IOCs from the data sources
which allows the tools to include lot of false positives during the process of IOCs
creation.

To overcome these limitations, we propose iGen, a novel approach to fully automate
the process of IOC generation and analysis. iGen consists of different modules like data acquisition module, malware analysis module, intelligent IOC extraction module and security rule generator module. All these modules are arranged in a work-flow to fully automate the IOC generation process.  Proposed approach is based on the idea that our model can understand English texts like human beings, and extract
the IOCs from the different data sources intelligently. Identification of the IOCs
is done on the basis of the syntax and semantics of the sentence as well as context
words (e.g., ``attacked'', ``suspicious'') present in the sentence which helps the approach
work on any kind of data source. Our proposed technique, first removes the
words with no contextual meaning like stop words and punctuations etc. Then using
the rest of the words in the sentence and output label (IOC or non-IOC sentence), our model intelligently learn to classify sentences into IOC and non-IOC sentences.
Once IOC sentences are identified using this Convolutional Neural Network (CNN)
based approach, next step is to identify the IOC tokens (like domains, IP, URL) in
the sentences. This CNN based classification model helps in removing false positives
(like IPs which are not malicious). Afterwards, IOCs extracted from different data
sources are correlated to find the links between thousands of apparently unrelated
attack instances, particularly infrastructures shared between them. Our approach
fully automates the process of IOC generation from gathering data from different
sources to creating rules (e.g. OpenIOC, snort rules, STIX rules) for deployment on
the security infrastructure.




\end{abstract}
