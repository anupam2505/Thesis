\chapter{DISCUSSION AND FUTURE WORK} \label{chap:discussion}
Our experiments show that iGen not only fully automated the process of cyber threat intelligence collection but also generates deployable security rules from the data. With a large amount of IOCs automatically recovered from the wild and converted into a machine-readable form, these can be quickly and effectively utilized to counter emerging threats. For example, knowing the IP of C\&C server from APT report and then finding the email associated with IP address from malware report can help us find the actor behind the attack. This will enable the defender to disable or block the servers associated with that email to stop future attacks. On the other hand, our current design is still preliminary. Here, we discuss the limitations of iGen and potential follow-up research.

\subsection{Limitations}
Although iGen extracts IOC with high accuracy, precision and recall, but it's necessary to recognize our system limitations. First, iGen assumes that all the IOCs mentioned in the article are part of some sentence. But there is a possibility that some of them are present in images (e.g., command prompt screenshots) or tables inserted into the article. Second, even though iGen detects IOCs with high accuracy. iGen still introduces some false discoveries and misses some IOCs. These problems mostly arises due to abnormal ways of writing the report. Typos in articles or reports effects the performance of iGen. For example, if one forgets to place a space after the period, a sentence becomes held with the follow-up one, which could cause an error in IOC sentence classification. Further efforts are desirable to better address these issues.